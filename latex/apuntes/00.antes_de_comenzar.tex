% \documentclass[10pt, twocolumn, a4paper]{article}
\documentclass[12pt,a4paper]{article}

\usepackage[backend=biber, style=ieee]{biblatex}                        % To include the bibliography
\usepackage[left=2cm, right=2cm, top=2.5cm, bottom=2.5cm]{geometry}     % To set the margins
\usepackage[noend]{algpseudocode}
\usepackage[table]{xcolor}                                              % For coloring cells

\usepackage{algorithm}                                                  % To include algorithms
\usepackage{amsfonts}                                                   % To include math fonts:ToggleTerm direction=float
\usepackage{amsmath}                                                    % To include Mathematic symbols
\usepackage{authblk}                                                    % To format author affiliations
\usepackage{caption}                                                    % For caption spacing
\usepackage{enumitem}                                                   % To customize lists (items like i, ii, iii, iv)
\usepackage{float}                                                      % To place figures where you want them
\usepackage{fancyhdr}                                                   % To customize headers and footers
\usepackage{graphicx}                                                   % To include images
\usepackage{hyperref}                                                   % To include hyperlinks
\usepackage{lipsum}                                                     % TODO: remove this
\usepackage{listings}                                                   % To include code
\usepackage{tabularx}                                                   % For equal-width columns
\usepackage{tcolorbox}                                                  % To make colored boxes
\usepackage{tikz}                                                       % To draw graphs
\usepackage{titlesec}                                                   % To format section titles
\usepackage{xcolor}                                                     % To define colors

\usetikzlibrary{graphs,graphs.standard}
\usetikzlibrary{positioning}

\addbibresource{./references.bib}

% Esto es para poder hacer cajitas de código con el fondo gris
\lstset{
    language=C++,
    basicstyle=\ttfamily\footnotesize,
    keywordstyle=\color{blue}\bfseries,
    stringstyle=\color{green!60!black},
    commentstyle=\color{gray},
    backgroundcolor=\color{gray!05},
    frame=single,
    numbers=left,
    numberstyle=\footnotesize,
    stepnumber=1,
    numbersep=10pt,
    tabsize=2,
    showstringspaces=false,
    captionpos=b,
    breaklines=true,
}

% Para poder hacer flechas
\usetikzlibrary{shapes, arrows}

% Sección de definiciones
\titleformat{\section}{\Large\bfseries}{\thesection}{1em}{}
\titleformat{\subsection}{\large\bfseries}{\thesubsection}{1em}{}

% Caja de colores
\definecolor{mint}{RGB}{202,251,202}
\definecolor{yellow}{RGB}{255,255,202}
\definecolor{red}{RGB}{255,202,202}

% Variables globales para el documento
\newcommand{\facultyname}{UP | Facultad de Ingeniería}
\newcommand{\coursename}{Programación con GPUs}
\newcommand{\currentsemester}{Segundo Semestre}
\newcommand{\currentyear}{2025}

% Esto es para poder agregar comentarios al código
\newcommand{\comentario}[1]{\textcolor{gray}{// #1}}

% Definir los encabezados y pies de página
\pagestyle{fancy}
\fancyhf{} % Borra encabezados y pies de página

% Header configuration
\fancyhead[R]{\includegraphics[height=50px]{../apuntes/images/logo_up.jpg}}
\renewcommand{\headrulewidth}{0.4pt} % Agregar línea debajo del encabezado
\setlength{\headheight}{50pt} % Ajustar la altura del encabezado

% Footer configuration
\fancyfoot[L]{\facultyname}
\fancyfoot[C]{\thepage}
\fancyfoot[R]{\coursename}
\renewcommand{\footrulewidth}{0.4pt} % Agregar línea arriba del pie de página

% Configuración de la primera página
\fancypagestyle{firstpage}{
  \fancyhf{} % Borra encabezados y pies de página
  \fancyfoot[C]{\thepage} % Número de página centrado
  \renewcommand{\headrulewidth}{0pt} % Sin línea en el encabezado
  \renewcommand{\footrulewidth}{0.4pt} % Sin línea en el pie de página
}

% Configuración de la primera página
\AtBeginDocument{\thispagestyle{firstpage}}


\begin{document}

\begin{center}
  \LARGE\textbf{Introducción a la Programación Paralela (CUDA)} \\
  \Large{Teorica 00 - Antes de Comenzar} \\
  \normalsize{\currentsemester, \currentyear} \\
  \vspace{1em}
  \hrule
\end{center}

\vspace{1em}

\setcounter{section}{1}

\subsection{Introducción}
\label{sec:set_up}

Antes de comenzar, vas a tener que configurar todo el entorno de trabajo para poder correr los ejemplos y hacer pruebas
locales en tu computadora. Para ello vas a necesitar tener instalado un cliente de \href{git}{https://git-scm.com/} para
poder clonar el repositorio de la materia y tener toda la información de la materia.

Primero deberás clonar el repositorio
\href{https://github.com/felixcuello/up-materias}{https://github.com/felixcuello/up-materias}. Allí encontrarás todo lo más
actualizado de la materia inluyendo las teóricas, prácticas, soluciones y el entorno de desarrollo CUDA. Te recomendamos
acceder a la URL y revisar los directorios para saber lo que está disponible.

\subsection{Entorno de desarrollo}

El entorno de desarrollo CUDA para la materia se puede utilizar:

\begin{itemize}
  \item \textbf{con Docker}: Recomendada, porque no requiere más que teneru \href{Docker
    Desktop}{https://www.docker.com/products/docker-desktop/} o
    \href{Colima}{https://github.com/abiosoft/colima), y poder utilizar el `Makefile` provisto por la materia (o sino
    utilizar los comandos de docker manualmente}
  \item \textbf{con instalación local}: Hacer una instalación local utilizando la guía oficial de \href{CUDA de
    Nvidia}{https://docs.nvidia.com/cuda/cuda-quick-start-guide/index.html}.
\end{itemize}

\subsubsection{Instalación con Docker}

Docker es una herramienta que permite crear, desplegar y ejecutar aplicaciones en contenedores. Un contenedor es una
unidad estandarizada que aísla una aplicación y todas sus dependencias (sistema de archivos, bibliotecas, etc.) para que
pueda ejecutarse de manera consistente en cualquier entorno. Esto significa que puedes ejecutar la misma aplicación en
diferentes sistemas operativos y bibliotecas instaladas en el sistema \textit{host}.

Además, el \textit{overhead} que genera Docker es mínimo ya que no hace una virtualización completa del sistema como
puede ser una máquina virtual, sino que utiliza la arquitectura subyacente del sistema operativo. Esto permite que los
contenedores sean más ligeros y rápidos, lógicamente, la desventaja es que al no virtualizar no se puede correr
cualquier sistema operativo en el contenedor.

Para correr el container de docker deberás instalar alguna de las alternativas para correr docker como son: \href{Docker
Desktop}{https://www.docker.com/products/docker-desktop/} o \href{Colima}{https://github.com/abiosoft/colima} para
correr el container de docker provisto por la cátedra.

\subsubsection{Makefile y comandos de docker}

Hemos provisto un archivo \texttt{Makefile} para poder levantar el container de docker con facilidad, sin embargo, es
posible que tu sistema no tenga el comando \texttt{make} instalado. Y nuevamente hay dos opciones aquí. Por un lado se
pueden correr los comandos de \texttt{docker} por separado, o se puede usar el comando \texttt{make} para correr las
opciones disponibles en el \texttt{Makefile}.

\textbf{\texttt{Makefile}}

Si tu computadora tiene instalado \texttt{make} (o si instalaste \texttt{make}) estarás en condiciones de ejecutar
\texttt{make} en la terminal.

\begin{lstlisting}
// Este header puede no estar disponible en tu sistema
// en ese caso no lo incluyas, e inclui cada header a mano (como siempre)
#include <bits/stdc++.h>

using namespace std;

int main() {
  // solucion

  return 0;
}
\end{lstlisting}

\subsection{Instalación local}

La instalación local no se explica porque ya está totalmente documentada en la documentación oficial de \href{CUDA de
Nvidia}{https://docs.nvidia.com/cuda/cuda-quick-start-guide/index.html}.

\newpage

\vspace{1em}

\printbibliography


\end{document}
