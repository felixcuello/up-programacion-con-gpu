% \documentclass[10pt, twocolumn, a4paper]{article}
\documentclass[12pt,a4paper]{article}

\usepackage[backend=biber, style=ieee]{biblatex}                        % To include the bibliography
\usepackage[left=2cm, right=2cm, top=2.5cm, bottom=2.5cm]{geometry}     % To set the margins
\usepackage[noend]{algpseudocode}
\usepackage[table]{xcolor}                                              % For coloring cells

\usepackage{algorithm}                                                  % To include algorithms
\usepackage{amsfonts}                                                   % To include math fonts:ToggleTerm direction=float
\usepackage{amsmath}                                                    % To include Mathematic symbols
\usepackage{authblk}                                                    % To format author affiliations
\usepackage{caption}                                                    % For caption spacing
\usepackage{enumitem}                                                   % To customize lists (items like i, ii, iii, iv)
\usepackage{float}                                                      % To place figures where you want them
\usepackage{fancyhdr}                                                   % To customize headers and footers
\usepackage{graphicx}                                                   % To include images
\usepackage{hyperref}                                                   % To include hyperlinks
\usepackage{lipsum}                                                     % TODO: remove this
\usepackage{listings}                                                   % To include code
\usepackage{tabularx}                                                   % For equal-width columns
\usepackage{tcolorbox}                                                  % To make colored boxes
\usepackage{tikz}                                                       % To draw graphs
\usepackage{titlesec}                                                   % To format section titles
\usepackage{xcolor}                                                     % To define colors

\usetikzlibrary{graphs,graphs.standard}
\usetikzlibrary{positioning}

\addbibresource{./references.bib}

% Esto es para poder hacer cajitas de código con el fondo gris
\lstset{
    language=C++,
    basicstyle=\ttfamily\footnotesize,
    keywordstyle=\color{blue}\bfseries,
    stringstyle=\color{green!60!black},
    commentstyle=\color{gray},
    backgroundcolor=\color{gray!05},
    frame=single,
    numbers=left,
    numberstyle=\footnotesize,
    stepnumber=1,
    numbersep=10pt,
    tabsize=2,
    showstringspaces=false,
    captionpos=b,
    breaklines=true,
}

% Para poder hacer flechas
\usetikzlibrary{shapes, arrows}

% Sección de definiciones
\titleformat{\section}{\Large\bfseries}{\thesection}{1em}{}
\titleformat{\subsection}{\large\bfseries}{\thesubsection}{1em}{}

% Caja de colores
\definecolor{mint}{RGB}{202,251,202}
\definecolor{yellow}{RGB}{255,255,202}
\definecolor{red}{RGB}{255,202,202}

% Variables globales para el documento
\newcommand{\facultyname}{UP | Facultad de Ingeniería}
\newcommand{\coursename}{Programación con GPUs}
\newcommand{\currentsemester}{Segundo Semestre}
\newcommand{\currentyear}{2025}

% Esto es para poder agregar comentarios al código
\newcommand{\comentario}[1]{\textcolor{gray}{// #1}}

% Definir los encabezados y pies de página
\pagestyle{fancy}
\fancyhf{} % Borra encabezados y pies de página

% Header configuration
\fancyhead[R]{\includegraphics[height=50px]{../apuntes/images/logo_up.jpg}}
\renewcommand{\headrulewidth}{0.4pt} % Agregar línea debajo del encabezado
\setlength{\headheight}{50pt} % Ajustar la altura del encabezado

% Footer configuration
\fancyfoot[L]{\facultyname}
\fancyfoot[C]{\thepage}
\fancyfoot[R]{\coursename}
\renewcommand{\footrulewidth}{0.4pt} % Agregar línea arriba del pie de página

% Configuración de la primera página
\fancypagestyle{firstpage}{
  \fancyhf{} % Borra encabezados y pies de página
  \fancyfoot[C]{\thepage} % Número de página centrado
  \renewcommand{\headrulewidth}{0pt} % Sin línea en el encabezado
  \renewcommand{\footrulewidth}{0.4pt} % Sin línea en el pie de página
}

% Configuración de la primera página
\AtBeginDocument{\thispagestyle{firstpage}}


\begin{document}

\begin{center}
    \LARGE\textbf{\coursename} \\
    \Large{Paralelización de algoritmos con CUDA} \\
    \normalsize{\currentsemester} \\
    \vspace{1em}
    \hrule
\end{center}

\vspace{1em}


%  FUNDAMENTACIÓN
% --------------------------------------------------------------------------------------------------
\section*{Fundamentación}
Los estudiantes de las carreras de Informática y afines de la Facultad de Ingeniería de la Universidad de Palermo, deben
adquirir conocimientos básicos de programación, como así también de ciencias de la computación a través de las materias
que componen el tronco común de la carrera como es el caso de "Introducción a la Programación", "Estructura de Datos y
Algoritmos", "Algoritmos 1" y "Algoritmos 2". Sin embargo, esas materias hacen foco en la programación secuencial, es
decir, la ejecución de un solo hilo de ejecución basado en CPUs ya que ya que estas han sido desarrolladas para
minimizar la latencia y maximizar el rendimiento de estas aplicaciones secuenciales. Sin embargo hay determinadas
aplicaciones y algoritmos que sólo pueden ser optimizados si se corren de manera paralela. Para ello se requieren tantos
CPUs como GPUs, pero se necesitan lenguajes de programación y técnicas de programación diferentes que las que se
utilizan en las CPUs.

Esta materia se enfoca en la programación paralela utilizando CUDA (Compute Unified Device Architecture), que es una
plataforma que permite la ejecución simultánea de tareas o procesos utilizando múltiples núcleos de procesamiento
gráfico (GPU) para realizar cálculos en paralelo permitiendo mejorar el rendimiento de sus aplicaciones. Esto
beneficiará a los estudiantes permitiéndoles adquirir habilidades prácticas en la comprensión de algoritmos paralelos y
la posibilidad de desarrollar aplicaciones que puedan procesar datos de manera masivamente paralela.


%  OBJETIVOS
% --------------------------------------------------------------------------------------------------
\section*{Objetivos}
\begin {itemize}
  \item Comprender los conceptos básicos de la programación paralela y su importancia en la computación moderna.
  \item Hacer análisis de los algoritmos paralelos y su aplicación en la resolución de problemas complejos.
  \item Aprender a utilizar CUDA para desarrollar aplicaciones paralelas en C.
  \item Investigar y aplicar las herramientas y bibliotecas disponibles para la programación paralela en CUDA.
  \item Desarrollar habilidades prácticas en la implementación de algoritmos paralelos utilizando CUDA.
  \item Explorar casos de uso y aplicaciones de la programación paralela en diferentes dominios.
  \item Evaluar el rendimiento y la escalabilidad de las aplicaciones paralelas desarrolladas con CUDA.
  \item Entender cómo se deben abordar los problemas de programación paralela
\end {itemize}


%  MODULO 01: INTRODUCCIÓN
% --------------------------------------------------------------------------------------------------
\subsection{Módulo 1: Introducción a la Programación Paralela}

\textbf{Límites del Paralelismo}: Complejidad polinomial, Clases P, Clases de Nick. \textbf{Introducción a la
Programación Paralela} Evolución de los microprocesadores y la escalabilidad vertical. Importancia del paralelismo
masivo y la elección de CUDA C. \textbf{Comparación entre CPUs y GPUs} Diferencias en optimización y rendimiento entre
CPUs y GPUs. Casos de uso y limitaciones de cada tipo de procesador. \textbf{Modelo de Programación CUDA} Descripción
del modelo de programación desarrollado por NVIDIA. Ventajas económicas y de rendimiento de utilizar GPUs con CUDA.
\textbf{Paralelización de Aplicaciones} Beneficios y limitaciones de la paralelización. Ley de Amdahl y su impacto en el
rendimiento de programas paralelos. \textbf{Desafíos en la Programación Paralela} Complejidad en el diseño de algoritmos
paralelos. Sensibilidad a los datos de entrada y límites de acceso a memoria.
\end{document}


%  MODULO 02: INTRODUCCIÓN A CUDA
% --------------------------------------------------------------------------------------------------
\subsection{Módulo 2: Introducción a CUDA}

\textbf{Estructura de Programas CUDA}: Modelo de programación heterogéneo con código para CPU (host) y GPU (device).
\textbf{Funciones Kernel y Threads}: Definición de funciones con \texttt{\_\_global\_\_}, \texttt{\_\_device\_\_} y
\texttt{\_\_host\_\_}. Organización de threads en bloques y grids para ejecución paralela. \textbf{Gestión de Memoria en
GPU}: Uso de \texttt{cudaMalloc}, \texttt{cudaFree} y \texttt{cudaMemcpy} para manejo de memoria global del device.
Transferencia de datos entre host y device. \textbf{Optimización de Algoritmos}: Patrones comunes de paralelismo como
reducción y loop-parallelism. \textbf{Aplicaciones Prácticas}: Implementación de operaciones vectoriales, procesamiento
de imágenes y otros algoritmos paralelizables. \textbf{Manejo de Errores}: Mejores prácticas para detectar y manejar
errores en código CUDA.


