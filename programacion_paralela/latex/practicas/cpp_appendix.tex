% \documentclass[10pt, twocolumn, a4paper]{article}
\documentclass[12pt,a4paper]{article}

\usepackage[backend=biber, style=ieee]{biblatex}                        % To include the bibliography
\usepackage[left=2cm, right=2cm, top=2.5cm, bottom=2.5cm]{geometry}     % To set the margins
\usepackage[noend]{algpseudocode}
\usepackage[table]{xcolor}                                              % For coloring cells

\usepackage{algorithm}                                                  % To include algorithms
\usepackage{amsfonts}                                                   % To include math fonts:ToggleTerm direction=float
\usepackage{amsmath}                                                    % To include Mathematic symbols
\usepackage{authblk}                                                    % To format author affiliations
\usepackage{caption}                                                    % For caption spacing
\usepackage{enumitem}                                                   % To customize lists (items like i, ii, iii, iv)
\usepackage{float}                                                      % To place figures where you want them
\usepackage{fancyhdr}                                                   % To customize headers and footers
\usepackage{graphicx}                                                   % To include images
\usepackage{hyperref}                                                   % To include hyperlinks
\usepackage{lipsum}                                                     % TODO: remove this
\usepackage{listings}                                                   % To include code
\usepackage{tabularx}                                                   % For equal-width columns
\usepackage{tcolorbox}                                                  % To make colored boxes
\usepackage{tikz}                                                       % To draw graphs
\usepackage{titlesec}                                                   % To format section titles
\usepackage{xcolor}                                                     % To define colors

\usetikzlibrary{graphs,graphs.standard}
\usetikzlibrary{positioning}

\addbibresource{./references.bib}

% Esto es para poder hacer cajitas de código con el fondo gris
\lstset{
    language=C++,
    basicstyle=\ttfamily\footnotesize,
    keywordstyle=\color{blue}\bfseries,
    stringstyle=\color{green!60!black},
    commentstyle=\color{gray},
    backgroundcolor=\color{gray!05},
    frame=single,
    numbers=left,
    numberstyle=\footnotesize,
    stepnumber=1,
    numbersep=10pt,
    tabsize=2,
    showstringspaces=false,
    captionpos=b,
    breaklines=true,
}

% Para poder hacer flechas
\usetikzlibrary{shapes, arrows}

% Sección de definiciones
\titleformat{\section}{\Large\bfseries}{\thesection}{1em}{}
\titleformat{\subsection}{\large\bfseries}{\thesubsection}{1em}{}

% Caja de colores
\definecolor{mint}{RGB}{202,251,202}
\definecolor{yellow}{RGB}{255,255,202}
\definecolor{red}{RGB}{255,202,202}

% Variables globales para el documento
\newcommand{\facultyname}{UP | Facultad de Ingeniería}
\newcommand{\coursename}{Programación con GPUs}
\newcommand{\currentsemester}{Segundo Semestre}
\newcommand{\currentyear}{2025}

% Esto es para poder agregar comentarios al código
\newcommand{\comentario}[1]{\textcolor{gray}{// #1}}

% Definir los encabezados y pies de página
\pagestyle{fancy}
\fancyhf{} % Borra encabezados y pies de página

% Header configuration
\fancyhead[R]{\includegraphics[height=50px]{../apuntes/images/logo_up.jpg}}
\renewcommand{\headrulewidth}{0.4pt} % Agregar línea debajo del encabezado
\setlength{\headheight}{50pt} % Ajustar la altura del encabezado

% Footer configuration
\fancyfoot[L]{\facultyname}
\fancyfoot[C]{\thepage}
\fancyfoot[R]{\coursename}
\renewcommand{\footrulewidth}{0.4pt} % Agregar línea arriba del pie de página

% Configuración de la primera página
\fancypagestyle{firstpage}{
  \fancyhf{} % Borra encabezados y pies de página
  \fancyfoot[C]{\thepage} % Número de página centrado
  \renewcommand{\headrulewidth}{0pt} % Sin línea en el encabezado
  \renewcommand{\footrulewidth}{0.4pt} % Sin línea en el pie de página
}

% Configuración de la primera página
\AtBeginDocument{\thispagestyle{firstpage}}


\begin{document}

\begin{center}
    \LARGE\textbf{Programación Paralela} \\
    \Large{Apéndice: C++} \\
    \normalsize{\currentsemester, \currentyear} \\
    \vspace{1em}
    \hrule
\end{center}

\section{Introducción al C y C++}

Aunque muchos de los ejemplos de CUDA estarán en C, podremos utilizar C++ para simplificar los ejemplos del host. ya que
C++ es un lenguaje que tiene una biblioteca estándar muy amplia y que es muy eficiente en términos de tiempo de
ejecución y para algunas cosas es más sencillo.

Se asume que tenés algún conocimiento básico de programación en lenguaje C (u otro lenguaje de alto nivel), con lo cual
no se explicarán conceptos como \texttt{if} o \texttt{int a = 0}, pero sí se explicarán conceptos que en C no se ven
como \texttt{vector} o \texttt{sort}.

\begin{lstlisting}
// Este header puede no estar disponible en tu sistema
// en ese caso no lo incluyas, e inclui cada header a mano (como siempre)
#include <bits/stdc++.h>

using namespace std;

int main() {
  // solucion

  return 0;
}
\end{lstlisting}

El código que se ejecuta comienza en la sección main, aunque se pueden definir funciones adicionales si es necesario.
Para compilarlo tendremos que utilizar el compilador de NVIDIA $nvcc$ con los siguientes parámetros:

\begin{lstlisting}
  nvcc codigo.cc -o binario
\end{lstlisting}

Donde \texttt{codigo.cc} es el nombre del archivo que contiene el código fuente y \texttt{binario} es el nombre
del ejecutable que se generará. Si querés utilizar C, sólo hay que cambiar la extensión del código fuente a
\texttt{codigo.c}.

% -----------------------------------------------------------------------------------------------
\subsection{Entrada y salida}

Si bien en C se aprendieron a utilizar las funciones \texttt{scanf} y \texttt{printf}, en C++ se utilizan los objetos
\texttt{cin} y \texttt{cout} para leer y escribir datos al flujo de entrada y salida (STDIN y STDOUT, respectivamente),
que son más fáciles de utilizar, ya que ignora los espacios y puede utilizarse como un \textit{parser} ejemplo:


% -----------------------------------------------------------------------------------------------
\subsubsection{cin}

\begin{lstlisting}
string pre, post;
int n;

cin >> pre >> n >> post;
\end{lstlisting}

Este código por ejemplo podría leer una patente argentina de las nuevas, independientemente de la cantidad de espacios
que haya entre las letras y los números (incluso puede funcionar si se ingresan \textit{new lines}).

\begin{lstlisting}
  AA 234 BB
\end{lstlisting}

\begin{lstlisting}
  AA
345
CD
\end{lstlisting}

% -----------------------------------------------------------------------------------------------
\subsubsection{cout}

Lo mismo se puede hacer con la salida, utilizando \texttt{cout}:

\begin{lstlisting}
string pre = "AA", post = "BB";
int n = 234;
cout << pre << " " << n << " " << post << endl;
\end{lstlisting}

Que imprimiría la patente \texttt{AA 234 BB}.


% -----------------------------------------------------------------------------------------------
\subsection{while(cin)}

Para leer una cantidad indefinida de datos, se puede utilizar el bucle \texttt{while(cin)}, de la siguiente forma:

\begin{lstlisting}
int n;
while(cin >> n) {
  cout << n << endl;
}
\end{lstlisting}

Este código leerá números enteros hasta que se llege al \texttt{EOF} (End Of File), que en la terminal.

% -----------------------------------------------------------------------------------------------
\subsection{for}

El lenguaje C++ ha mejorado la forma de hacer bucles \texttt{for}, permitiendo hacer bucles más sencillos utilizando la
palabra reservada \texttt{auto} para inferir el tipo de dato de la variable que se está iterando. Por ejemplo:

\begin{lstlisting}
  vector<int> v = {1, 2, 3, 4, 5};
  for(auto i : v) {
    cout << i << endl;
  }
\end{lstlisting}

Que facilita muchísimo la escritura del código.


% -----------------------------------------------------------------------------------------------
\subsection{getline}

Puede ser necesario tener que leer una línea completa y no ignorar los espacios como cuando utilizamos \texttt{cin}, en
este caso se puede utilizar \texttt{getline} de la siguiente manera:

\begin{lstlisting}
string s;
getline(cin, s);
\end{lstlisting}


% -----------------------------------------------------------------------------------------------
\section{int, long long, string}
\label{sec:tiposdedatos}

En C++ hay varios tipos de datos básicos (o no tanto) que hay que saber utilizar. Los más comunes son \texttt{int},
\texttt{long long} y \texttt{string}. El primero es un entero de 32 bits, el segundo es un entero de 64 bits y el
tercero es una cadena de caracteres (ambos pueden ser signed or \texttt{unsigned}).

Si bien se suelen utilizar \texttt{int} en la mayoría de los problemas, hay que tener en cuenta que puede haber
problemas en los cuales elegir signo o tamaño incorrecto puede llevar a errores en la solución. Por ejmeplo, un error
común suele ser hacer esto:

\begin{lstlisting}
int a = 123456789;
long long b = a * a; // b = -1757895751
\end{lstlisting}

Esto se da porque \texttt{a * a} se calcula como un \texttt{int} y luego se convierte a \texttt{long long}, lo que hace
que por un momento \texttt{a * a} sea un \textit{overflow} de \texttt{int}, y por ende el resultado sea un número
negativo.


% -----------------------------------------------------------------------------------------------
\section{Vectores (arrays dinámicos)}

Los vectores en C++ son una estructura de datos similar a los arrays de C pero con algunas diferencias. La principal
diferencia es que los vectores pueden cambiar de tamaño dinámicamente, es decir, que se pueden agregar o quitar
elementos en tiempo de ejecución. Es importante señalar que los vectores en C++ comienzan en la posición 0, al igual que
los arrays de C, y que se accede a los elementos de la misma forma que en C. Por otra parte la complejidad de las
operaciones de acceso son $O(1)$ al igual que en C y agregar o quitar elementos al final del vector es $O(1)$ en
promedio (ya que en el peor caso puede llegar a ser $O(n)$). Sin embargo agregar o quitar elementos en cualquier otra
posición es $O(n)$.

La forma de declarar un vector en C++ es la siguiente:

\begin{lstlisting}
vector<T> v;
\end{lstlisting}

Donde \texttt{T} es el tipo de dato que se almacenará en el vector. Por ejemplo, si se quiere un vector de enteros, se
puede hacer lo siguiente:

\begin{lstlisting}
vector<int> v;
\end{lstlisting}

Mientras que se puede inicializar un vector con un tamaño específico de la siguiente forma:

\begin{lstlisting}
vector<int> v(10);
\end{lstlisting}

En este caso el vector \texttt{v} tendrá un tamaño de 10 elementos (normalmente inicializados en 0). También se puede
inicializar un vector con un tamaño específico y un valor específico de la siguiente forma:

\begin{lstlisting}
vector<int> v(10, 5);
\end{lstlisting}

En este caso el vector \texttt{v} tendrá un tamaño de 10 elementos, todos inicializados en 5. Para acceder a los
elementos de un vector se puede hacer de la misma forma que en un array de C, por ejemplo:

\begin{lstlisting}
vector<int> v(10, 5);
v[0] += 1;
cout << v[0] << endl; // Imprime 6
\end{lstlisting}

Para agregar un elemento al final de un vector se puede hacer de la siguiente forma:

\begin{lstlisting}
vector<int> v;
v.push_back(5);
\end{lstlisting}

En este caso el vector \texttt{v} tendrá un tamaño de 1 elemento, y el elemento en la posición 0 será 5. Para quitar un
elemento al final de un vector se puede hacer de la siguiente forma:

\begin{lstlisting}
vector<int> v;
v.push_back(5);
v.pop_back();
\end{lstlisting}

En este caso el vector \texttt{v} tendrá un tamaño de 0 elementos. Para recorrer un vector se puede hacer de la
siguiente forma:

\begin{lstlisting}
vector<int> v(10, 5);
for(int i = 0; i < v.size(); i++) {
  cout << v[i] << endl;
}
\end{lstlisting}

En este caso se recorre el vector \texttt{v} e imprime cada uno de los elementos.

\subsection{Iteradores}

Para recorrer un vector de forma más sencilla se puede utilizar un iterador, los iteradores son similares a los punteros
en C y tienen la ventaja de que el iterador se puede mover tanto hacia adelante como hacia atrás con el operador
\texttt{++} y \texttt{--}, respectivamente. Por ejemplo:

\begin{lstlisting}
vector<int> v(10, 5);
for(auto it = v.begin(); it != v.end(); it++) {
  cout << *it << endl;
}
\end{lstlisting}

\subsection{algorithms}

\subsubsection{sort}

Además de los iteradores, los vectores tienen una serie de funciones que permiten realizar operaciones sobre ellos. Por
ejemplo, se puede ordenar un vector de la siguiente forma:

\begin{lstlisting}
#include <algorithm>

vector<int> v = {3, 1, 4, 1, 5, 9, 2, 6, 5, 3, 5};
sort(v.begin(), v.end());
\end{lstlisting}

En este caso el vector \texttt{v} quedará ordenado de menor a mayor entre los índices \texttt{v.begin()} y
\texttt{v.end()}, ya que \texttt{v.begin()} es un iterador que apunta al primer elemento del vector y \texttt{v.end()}
es un iterador que apunta al final del vector (NO apunta al último elemento del vector, ya que \texttt{v.begin()} sería
la posición $0$ y \texttt{v.end()} sería la posición $n$).

Cabe señalar que la función \texttt{sort} es una función de la librería \texttt{algorithm} y que se puede utilizar para
ordenar cualquier tipo de contenedor que tenga iteradores, no sólo vectores, y además ordena entre cualquier rango de
iteradores, no sólo entre \texttt{begin()} y \texttt{end()}. Por ejemplo podríamos ordenar los elementos entre los
índices 2 y 5 de un vector de la siguiente forma:

\begin{lstlisting}
sort(v.begin() + 2, v.begin() + 5);
\end{lstlisting}

Nótese que \texttt{v.begin() + 2} es un iterador que apunta al tercer elemento del vector, ya que el operador \texttt{+}
retorna otro iterador que apunta 2 posiciones más adelante que el iterador original. Lo mismo sucede con \texttt{-} que
retorna un iterador que apunta a una posición anterior, por ejemplo \texttt{v.end() - 1} es un iterador que apunta al
último elemento del vector, ya que \texttt{v.end()} apunta a una posición después del último elemento del vector.


\subsubsection{reverse sort}

En C++ ordenar un vector de mayor a menor es un poco más complicado que ordenarlo de menor a mayor, ya que la función
\texttt{sort} no tiene un parámetro que permita ordenar de mayor a menor, para ello se utilizan iteradores reversos que
se mueven de atrás hacia adelante. Por ejemplo:

\begin{lstlisting}
sort(v.rbegin(), v.rend());
\end{lstlisting}

En este caso el vector \texttt{v} quedará ordenado de mayor a menor entre los índices \texttt{v.rbegin()} y
\texttt{v.rend()}.

\newpage

\vspace{1em}

\printbibliography


\end{document}
