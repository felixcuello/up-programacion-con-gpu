% \documentclass[10pt, twocolumn, a4paper]{article}
\documentclass[12pt,a4paper]{article}

\usepackage[backend=biber, style=ieee]{biblatex}                        % To include the bibliography
\usepackage[left=2cm, right=2cm, top=2.5cm, bottom=2.5cm]{geometry}     % To set the margins
\usepackage[noend]{algpseudocode}
\usepackage[table]{xcolor}                                              % For coloring cells

\usepackage{algorithm}                                                  % To include algorithms
\usepackage{amsfonts}                                                   % To include math fonts:ToggleTerm direction=float
\usepackage{amsmath}                                                    % To include Mathematic symbols
\usepackage{authblk}                                                    % To format author affiliations
\usepackage{caption}                                                    % For caption spacing
\usepackage{enumitem}                                                   % To customize lists (items like i, ii, iii, iv)
\usepackage{float}                                                      % To place figures where you want them
\usepackage{fancyhdr}                                                   % To customize headers and footers
\usepackage{graphicx}                                                   % To include images
\usepackage{hyperref}                                                   % To include hyperlinks
\usepackage{lipsum}                                                     % TODO: remove this
\usepackage{listings}                                                   % To include code
\usepackage{tabularx}                                                   % For equal-width columns
\usepackage{tcolorbox}                                                  % To make colored boxes
\usepackage{tikz}                                                       % To draw graphs
\usepackage{titlesec}                                                   % To format section titles
\usepackage{xcolor}                                                     % To define colors

\usetikzlibrary{graphs,graphs.standard}
\usetikzlibrary{positioning}

\addbibresource{./references.bib}

% Esto es para poder hacer cajitas de código con el fondo gris
\lstset{
    language=C++,
    basicstyle=\ttfamily\footnotesize,
    keywordstyle=\color{blue}\bfseries,
    stringstyle=\color{green!60!black},
    commentstyle=\color{gray},
    backgroundcolor=\color{gray!05},
    frame=single,
    numbers=left,
    numberstyle=\footnotesize,
    stepnumber=1,
    numbersep=10pt,
    tabsize=2,
    showstringspaces=false,
    captionpos=b,
    breaklines=true,
}

% Para poder hacer flechas
\usetikzlibrary{shapes, arrows}

% Sección de definiciones
\titleformat{\section}{\Large\bfseries}{\thesection}{1em}{}
\titleformat{\subsection}{\large\bfseries}{\thesubsection}{1em}{}

% Caja de colores
\definecolor{mint}{RGB}{202,251,202}
\definecolor{yellow}{RGB}{255,255,202}
\definecolor{red}{RGB}{255,202,202}

% Variables globales para el documento
\newcommand{\facultyname}{UP | Facultad de Ingeniería}
\newcommand{\coursename}{Programación con GPUs}
\newcommand{\currentsemester}{Segundo Semestre}
\newcommand{\currentyear}{2025}

% Esto es para poder agregar comentarios al código
\newcommand{\comentario}[1]{\textcolor{gray}{// #1}}

% Definir los encabezados y pies de página
\pagestyle{fancy}
\fancyhf{} % Borra encabezados y pies de página

% Header configuration
\fancyhead[R]{\includegraphics[height=50px]{../apuntes/images/logo_up.jpg}}
\renewcommand{\headrulewidth}{0.4pt} % Agregar línea debajo del encabezado
\setlength{\headheight}{50pt} % Ajustar la altura del encabezado

% Footer configuration
\fancyfoot[L]{\facultyname}
\fancyfoot[C]{\thepage}
\fancyfoot[R]{\coursename}
\renewcommand{\footrulewidth}{0.4pt} % Agregar línea arriba del pie de página

% Configuración de la primera página
\fancypagestyle{firstpage}{
  \fancyhf{} % Borra encabezados y pies de página
  \fancyfoot[C]{\thepage} % Número de página centrado
  \renewcommand{\headrulewidth}{0pt} % Sin línea en el encabezado
  \renewcommand{\footrulewidth}{0.4pt} % Sin línea en el pie de página
}

% Configuración de la primera página
\AtBeginDocument{\thispagestyle{firstpage}}


\begin{document}

\begin{center}
    \LARGE\textbf{Programación Paralela} \\
    \Large{Teórica 02 - Introducción a CUDA} \\
    \normalsize{\currentsemester, \currentyear} \\
    \vspace{1em}
    \hrule
\end{center}

\setcounter{section}{2}

\subsection{Introducción}

Como dijimos en la práctica anterior, la computación paralela es una técnica que permite realizar cálculos de manera
simultánea en múltiples procesadores. Para los ejercicios vamos a utilizar CUDA C que nos permite escribir programas
paralelos escalables en sistemas donde conviven tanto CPUs como GPUs. \textbf{En este tipo de sistemas donde hay
porciones de código que pueden ejecutarse en paralelo, pero que están gobernadas por un código secuencial que corre en
la CPU, los llamaremos \textit{sistemas heterogéneos}}. CUDA extiende el lenguaje de programación C con una sintaxis
mínima que permite tener código secuencial, gobernado por la CPU, y código que puede ser ejecutado en paralelo en GPUs.

Cuando en el software moderno las aplicaciones se ejecutan \textit{lento}, el problema usualmente suele ser que hay
demasiados datos para ser procesados. Por ejemplo en el procesamiento de imágenes o videos, la simulación de dinámica de
fluidos, el manejo de sistemas complejos como (líneas aéreas), o incluso cosas mucho más sencillas como convertir una
imagen de pixels a escala de grises. Estas tareas se pueden fraccionar en tareas más pequeñas que pueden ser ejecutadas
de manera independiente y paralela.

\begin{tcolorbox}[colback=mint,colframe=yellow!75!black,arc=0pt,outer arc=0pt]
  \textbf{Paralelismo de tareas vs. paralelismo de datos} \\

  El paralelismo de tareas, se refiere a la ejecución de múltiples tareas (no necesariamente las mismas) al mismo tiempo.
  Mientras que el paralelismo de datos, se refiere a la ejecución de la misma tarea con datos diferentes al mismo tiempo.
\end{tcolorbox}


\subsection{Convertir una Imagen a Escala de Grises}

El procesamiento de imágenes es un clásico ejemplo de computación paralela y comenzaremos la introducción a CUDA viendo
un ejemplo de procesamiento de imágenes para ejemplificar los primeros conceptos. En este caso vamos a ver cómo
convertir una imagen a escala de grises.

\subsubsection{Representación de una imagen en la computadora}

Antes de comenzar a escribir cualquier tipo de código, tenemos que saber cómo se representa una imagen en la
computadora. Esencialmente una imagen se representa como una matriz de tuplas $(R, G, B)$ donde $R$, $G$ y $B$ son los
valores de los colores (canales) rojo, verde y azul respectivamente. Cada uno de estos canales tiene un valor que va
desde 0 ($0x00$) a 255 ($0xFF$) y representan la intensidad de cada color de un pixel en una imagen. Existen $256^3$
colores posibles y estos colores se representan dentro del triángulo AdobeRGB (ver Figura \ref{fig:adobe_rgb}).

\begin{figure}[H]
  \centering
  \includegraphics[width=200px]{./images/adobe_rgb.png}
  \caption{Triángulo AdobeRGB}
  \label{fig:adobe_rgb}
\end{figure}

Como dijimos, estos valores de $R$, $G$ y $B$ representan los \textit{canales} y para convertir estos canales a escala
de grises se debe utilizar una fórmula, ya que hay que decidir cuál va a ser la intensidad final del pixel en escala de
grises en la imagen blanco y negro. Esto se hace realizando una combinación lineal de los valores de los canales de
color de alguna manera y, lógicamente hay muchas formas de convertir estos canales a escala de grises. Por ejemplo
podríamos tomar sólo el valor del canal $G$ (verde), haciendo la combinación lineal $pixel_{gris} = R \cdot 0 + G \cdot
1 + B \cdot 0$. De esta manera la imagen sólo se vería representada por ese canal, lo cual no sería muy verídico,
ya que el ojo humano percibe la intensidad de los 3 canales. Como podrán imaginar, hay muchas formas de convertir una
imagen a escala de grises \footnote{¡Podrían probar con diferentes fórmulas y ver cómo se ve la imagen!, hay $256^3$
combinaciones posibles.}, pero hay algunas que son más comunes que otras.

\begin{tcolorbox}[colback=mint,colframe=yellow!75!black,arc=0pt,outer arc=0pt]
  \textbf{Algunas fórmulas comunes para convertir a escala de grises}, son las siguientes: \\

  \begin{itemize}
    \item \textbf{Promedio}: $I = \frac{R + G + B}{3}$
    \item \textbf{Luminosidad}: $I = 0.21R + 0.72G + 0.07B$
    \item \textbf{Desaturación}: $I = \frac{max(R, G, B) + min(R, G, B)}{2}$
  \end{itemize}
\end{tcolorbox}

Para nuestro ejemplo, vamos a usar la fórmula de \textit{luminosidad} para convertir la imagen a escala de grises que es
un promedio pesado de los canales de color, y que representa la percepción humana de la luminosidad, pero como dijimos
no es la única forma y cambiar esta fórmula puede dar diferentes resultados que podrían considerarse como "filtros de
imagen".

\subsubsection{Estructura de un programa en CUDA C}

La estructura de un programa en CUDA C es similar a un programa en C, lo que refleja su naturaleza heterogénea donde
existe un \textit{host} (CPU) y uno o más \textit{devices} (GPUs) en la computadora. El código fuente CUDA tiene una
mezcla de ambos códigos, uno que se ejecuta en el host, y otro en los devices. Por defecto, todo el código se ejecuta en
el host, aunque vamos a declarar funciones de una forma especial para que puedan ser corridas en los \textit{devices}.

El código con estas extensiones de CUDA tiene que ser compilado con el compilador de NVIDIA, \texttt{nvcc}, que es un
wrapper para el compilador de C, \texttt{gcc}. El compilador de NVIDIA se encarga de: separar el código que se ejecuta en
el host, separar el código que se ejecuta en el device, y de compilarlo con el compilador de C. \textbf{El código
identificado con las \textit{keywords} (palabras reservadas) de CUDA para las funciones paralelas se denominan
\textit{kernels}}. Estos kernels son funciones que están asociadas a estructuras de datos y que van a ser ejecutadas en
paralelo por GPUs. En las situaciones donde no haya una GPU disponible, el código de todas formas se ejecutará en una
CPU (uno podría incluso ejecuar el kernel en una CPU utilizando herramientas como MCUDA) \cite{sutter2005}.

La ejecuciónm de un programa en CUDA se ilustra en la Figura \ref{fig:execution_cuda}, la ejecución comienza con el
código del host (CPU) y cuando se llama a una función kernel (código paralelo del dispositivo) es ejecutada por un gran
número de threads en un dispositivo. Todos los threads que son generados por un kernel son colectivamente llamados un
\textit{grid}. Estos threads son el vehículo principal de la ejecución paralela en una plataforma CUDA. Cuando todos los
threads de un kernel completan su ejecución, el grid correspondiente termina, y la ejecución continúa en el host hasta
que otro kernel es lanzado. Notar que la Figura \ref{fig:execution_cuda} muestra un modelo simplificado donde la
ejecución del código de la CPU y la GPU no se superponen, pero en muchas de las aplicaciones heterogéneas, la CPU y las
GPUs para aprovechar al máximo ambos recursos.

\begin{figure}[H]
  \centering
  \includegraphics[width=400px]{./images/execution_cuda.png}
  \caption{Ejecución de un programa en CUDA}
  \label{fig:execution_cuda}
\end{figure}

\begin{tcolorbox}[colback=yellow,colframe=yellow!75!black,arc=0pt,outer arc=0pt]
  \textbf{Threads} \\

  La ejecución de un kernel genera un gran número de threads para explotar el paralelismo de datos. En el caso de la
  conversión de una imagen color a escala de grises cada thread podría ser utilizado para computar un pixel de la imagen
  de salida. En este caso, el número de threads que serán generados por el kernel es igual al número de pixels en la
  imagen. Para imágenes grandes, un gran número de threads serán generados. En la práctica, cada thread puede procesar
  múltiples pixels para eficiencia. Los programadores de CUDA pueden asumir que estos threads toman muy pocos ciclos de
  reloj para ser generados y programados debido al soporte eficiente del hardware. Esto es en contraste con los threads
  tradicionales de la CPU que típicamente toman miles de ciclos de reloj para ser generados y programados.
\end{tcolorbox}

\subsection{}

\newpage

\vspace{1em}

\printbibliography


\end{document}
